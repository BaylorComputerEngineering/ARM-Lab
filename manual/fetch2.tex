\chapter{Program Counter Incrementer and Mux}

As mentioned in the last lab, the program counter is a register that is one word in length.  It holds the address in memory of the next instruction to be fetched and executed.  There are several ways that the program counter is updated:  
\begin{enumerate}
	\item If the program does not branch (via an if statement, while loop, etc), then the program counter should advance to the next address (by adding 4 to the current PC) each clock cycle.
	\item If the conditions of a conditional branch are met, then the program counter should be updated with the branch destination address.
	\item If an unconditional branch or jump occurs, then the program counter should be updated with the branch destination address.
	\item If an interrupt or error occurs, then the program counter should be updated with the interrupt or error handler address.
\end{enumerate}
The instructions will be fetched in sequential order the majority of the time.

\section{Incrementer}

We will build a program counter incrementer by making a simple adder.  Later in our computer we will need another adder, so we will re-use this code.  When used as the program counter, we will pass it a 4 because each instruction is 32-bits long (even though it is a 64-bit computer) and we want to increment to the next instruction in memory.  Most machines are byte addressable, because one ASCII character (a char in c/c++) is a byte.  For a machine with 32-bit instructions like we are using, that would mean that each instruction would be 4 bytes later in memory ($32/8=4$ bytes).  Therefore, we will be adding 4 to the program counter each time we want to increment the program counter.

An adder is very simple in Verilog.  There are two inputs (the two numbers to be added) and one output (the result).   All the ports are size word because they hold integers.  

In this lab you will make your own adder module.  Your adder module should be called 'adder' and should have inputs of \verb1a_in1 and \verb1b_in1.  The output should be \verb1add_out1.  HINT: this should be very easy.  Verilog is a Hardware Description Language, so use Verilog to describe what you want to do.  Don't make it complicated.  The adder code should be stored in ARM-Lab/code/0\_common/adder.v.  You will need to create this file.

\section{Input Selection via Mux}

We will also need to be able to choose between normal advancing (sequential stepping) and branching (loops, if statements, etc.).  We will use a multiplexor (mux) to do this.  A mux is a simple device that connects one of the inputs to the output based on how the control bit is set.  If the control bit is 0 then input a is connected to the output, and if the selector is 1 then input b is connected to the output.  One interesting addition in this block of code is the addition of a size parameter.  Parameters are passed before the normal ports and are used to configure the code to meet a requirement at the time of construction.  Note parameters are constants and cannot be changed later in the module.  The $=8$ defines the default value if nothing is specified.  In this case we are using parameters to set the number of wires that compose the inputs and output.  In our lab project, we will need some muxes to switch entire words (64 bits), but later we will also need to switch register addresses (5 bits).  Rather than write two muxes, we will make one and then use the parameter to change the size when they are declared.  The mux code should be stored in ARM-Lab/code/0\_common/mux.v.  Please look at the starter code in this lab document for direction on how to add a parameter to the module.

\Verilog{Verilog code to make a mux.}{code:mux}{../code/0_common/mux.v}

Look at the testbench provided for the mux.  Note that if the parameter is not set by the testbench, the mux module will set the inputs and outputs to be the default of 8.  We are going to change this to test it as a 64 bit mux and a 5-bit mux.  Notice how the size of the mux is set, since you will need to do this in future labs.

\section{Your Assignment}

You are to:
\begin{enumerate}
\item Create an adder module.
\item Use the provided adder\_test.sv to verify that the adder works properly.  Note that you cannot/should not make any changes to the test bench.  The correct results are already in the test bench.	
\item Create a mux module.
\item Use the provided mux\_test.sv to verify that the mux works properly.  Note that you cannot/should not make any changes to the test bench.  The correct results are already in the test bench.	
\item Rather than writing a lab report, please produce a landscape mode single page PDF called Lab2\_lastname.pdf that includes (in this order):
\begin{enumerate}
	\item Your name and the lab number.
	\item A snip of the Simulation Results for the adder.  Make sure to show all values in decimal form and don't cut off the signal names on the left.  
	\item A snip of the test results from the Tcl Console for the adder.  This snip should show the entire log from BEGIN TEST RESULTS to END TEST RESULTS.
	\item A snip of the Simulation Results for the mux.  Make sure to show all values in decimal form and don't cut off the signal names on the left.  
	\item A snip of the test results from the Tcl Console for the mux.  This snip should show the entire log from BEGIN TEST RESULTS to END TEST RESULTS.	
\end{enumerate}
\item Upload Lab2\_lastname.pdf file to Canvas.
\item Zip up your ARM-Lab directory and submit it on Canvas as well.  I will run your code against my correct testbench to verify that your code and testbench work correctly. 
\end{enumerate} 