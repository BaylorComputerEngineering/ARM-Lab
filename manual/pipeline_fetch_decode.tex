\chapter{Full Datapath}


\section{Datapath}
Now that we have a working single-cycle datapath, we are going to break it apart and start pipelining it.  Make sure to keep a copy of your working single cycle datapath for reference.  To begin the pipelining process, we will first pipeline the iFetch and iDecode stages.

The advantage of pipelining includes the ability to reduce the clock cycle time by doing only one stage at a time.  We have been using a 10ns clock cycle.  Now that we are pipelining, let's start conservative and go down to a 4ns clock cycle.  Please keep it at 4ns for now to maintain consistency across the class.  

This will affect your timing and will require you to update your delays in your datapath.  Ask yourself why you made each delay.  Was it really meant to be relative to the clock cycle time, or was it meant to be a constant number. For instance, if you put in a CYCLE/10 to give a register time to update, do you really want that time to drop from 1ns to 0.4ns?  Has anything about the register update time changed?  The answer is no.  So you need to be strategic with your delays throughout these two stages.  You might also need to modify your delay functions.  The goal for today is to get a series of instructions to execute in pipeline form, where a new instruction is fetched while the previous instruction is being decoded.  You should verify this by examining the Read\_Register values, Read\_Data values, control signal values, etc.  The next step will be buffering nPC and the instruction, but we will focus on that next class.

To test this, use your set of instructions that corresponds to your Expected Results table.

\section{Your Assignment}

You are to:
\begin{enumerate}
\item Comment out Execute, Memory, and WriteBack stages from datapath.v.
\item Pipeline iFetch and iDecode stages and update delays.   
\item Test with your set of instructions from your Expected Results table.
\item Verify that stages are processing the correct instructions at the correct times.
\item There is no lab report today.
\end{enumerate} 