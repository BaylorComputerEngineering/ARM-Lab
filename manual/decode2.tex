\chapter{Control Unit and Sign Extender}

\section{Control Unit}
Next, we will create the main control unit.  You will create a new module called control (in control.v), and this module will be part of your iDecode module.  The control module should use a portion of the instruction to determine the values of all control signals to be used in our processor.  These signals include:
\begin{enumerate}
\item Reg2Loc
\item Uncondbranch
\item Branch
\item MemRead
\item MemtoReg
\item ALUOp
\item MemWrite
\item ALUSrc
\item RegWrite
\end{enumerate}

The supported instructions should include:
\begin{enumerate}
	\item ADD
	\item SUB
	\item AND
	\item ORR
	\item LDUR
	\item STUR
	\item CBZ
	\item B
\end{enumerate}

You will need to evaluate the incoming instruction and determine what value to set for each control line.  Inputs and outputs should match Figure 4.22 in the textbook.  Note that a value of X in a table entry indicates that it does not matter whether the value is 0 or 1.  It is always best to set the signal to 0 or 1, even if it does not matter.

\section{Control Unit Test}
The Control Unit is crucial to operation of your datapath, so it needs to be tested thoroughly.  Every supported instruction (listed above) should be tested with the Control Unit Test.  To facilitate testing, I have provided an Excel spreadsheet with 10 instructions listed across the top.  I will refer to this table as the Expected Results Table.  These are the instructions that you should use to test your Control Unit and your full Decode stage (once we finish it).  The rows of the table are signals that will be inputs and outputs of the control module.  

Before writing your unit test, you should fill in every cell in this table with the expected results.  This includes producing the machine code for each instruction.  Then, use these machine code instructions as inputs to your Control Unit Test and verify the outputs.  You should display the outputs on your simulation results in the same order that they are shown on the Expected Results Table.  Then, you can go right down the table and easily verify that your results are correct.  Save the Expected Results Table on GitHub, because we will be adding to it in future labs.

Note that, at this point, the values that are loaded in the registers do not matter, as we are just testing the Control Unit.  Once we get to the full iDecode stage, we will need to address the register values.

The Verilog implementation between 'if/else' statements and 'case' statements differ.  'If/else' statements will create a series of nested muxes with two inputs each, whereas a case statement will produce one large mux with many inputs.  To maximize speed, we should use case statements.  But one of the challenges of this lab is dealing with opcodes of different sizes.  Thankfully, Verilog has 'casex', which will only evaluate binary digits that are not labeled X.  So for a CBZ instruction, you can fill in the last 3 digits with XXX and use 'casex'.  To help you get started, I've included starter code for control.v in  Listing~\ref{code:control}.  The 'casex' syntax will also be very helpful in the Sign Extender.  Also, please create macros in definitions.vh for the opcodes of each instruction and for the ALU Op values of each instruction type.

\Verilog{Control Module}{code:control}{../code/2_decode/control.v}

\section{Sign Extender}
The final major component of the Decode stage is the Sign Extender.  The Sign Extender should use information in the instruction to create a 64-bit output value to use as an address or branch offset.  The sign extender should support extending address values from the following instructions:
\begin{enumerate}
	\item LDUR
	\item STUR
	\item CBZ
	\item B
\end{enumerate}

\section{Sign Extender}
The Sign Extender Test should test LDUR, STUR, CBZ, and B instructions.  You can utilize the instructions from the Control Unit test.  Verify that you Sign Extended Output on the Simulation Results matches the value on your Expected Results table.

\clearpage
\section{Your Assignment}

You are to:
\begin{enumerate}
\item Fill in the entire Expected Results Table, starting with the version that I provide in the testfiles section in GitHub.
\item Create a Control Unit.
\item Create a Control Unit Test and verify that the simulation results match your Expected Results Table.
\item Create a Sign Extender module.
\item Create a Sign Extender Test Module and verify that the simulation results match your Expected Results Table.
\item We will not write a full lab report for this lab.  Instead, we will submit a simplified report that will just allow me to make sure that everyone is making good progress.  Write a simplified lab report in LaTex that includes:
\begin{enumerate}
	\item  A one paragraph intro describing your results.  Please indicate whether the modules worked properly or not.  If you have any other comments that you want me to know, please put them here.
	\item Your completed Expected Results Table.
	\item Your module code, test bench code, and simulation results for each module created.  Please make sure to label which set of simulation results goes with each module.  You do not need to add additional text in this section unless there is something that you want to communicate to me.
\end{enumerate} 
\end{enumerate} 