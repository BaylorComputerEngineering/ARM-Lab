\chapter{ALU and ALU Control}

\section{ALU}

First we will build the ALU itself.  The ALU has three inputs (two data inputs to act on, and a control input to determine the action perfomed) and two outputs (one data, and a logical flag). In the table shown in the lecture slides, you can see the meaning of the control bits used to determine what the ALU will calculate.  You should use a case statement for the control bits to determine which ALU operation to perform.  For each operation, you do not have to do anything fancy.  You just need to use the math capability that verilog provides to make the calculation.  You should give the the ALU control bits names in the definitions.vh file, and you should use these in your cases.  Also don't forget to make a default case, which is needed to actually wire this up.  Pick something fast for the default, thus usually a logic statement.

One last thing to note is the generation of the zero flag.  There are several ways to handle this, but the easiest way to handle it is below: 
\begin{enumerate}
\item In Verilog (like C), the statement $(y==0)$ is an operation with a boolean output.  You can thus say $x=(y==0);$ to assign $x$ to be the boolean value.  The statement $x=(y==0);$ is realizable as a digital comparator with $y$ and $0$ as inputs and $x$ as the single bit output.
\end{enumerate}

\section{ALU Control}

Now we need to build the controller to use the ALUOp field and the opcode field to generate the ALU control bits used above.  Consider the table for ALU Op to ALU Control bits in the lecture slides.  The ALU Control module is where you make this translation so that the ALU is told the proper operation to execute.  Use case statements and include a default to handle undefined signals (use fast commands for undefined signals).  Define the ALU operations in definitions.vh file to improve readability - you shouldn't need any numbers.  This should be a simple module with two inputs (ALUOp and function) and one output (control bits).

\section{Your Assignment}

You are to:
\begin{enumerate}
\item Create the ALU module.
\item Create an ALU testbench to test all relevant ALU operations, including testing the zero bit.
\item Create an ALU Control module.
\item Create an ALU Control testbench to test all of the opcode/ALUOp pairs that you will be using
\item You do not need to create Expected Results tables for these tests, as they are very short and simple.
\item Create a lab report in the LabN format.
\end{enumerate} 